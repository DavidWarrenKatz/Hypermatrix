\documentclass[12pt]{article}
\usepackage{amsmath, amssymb}
\usepackage{graphicx}
\usepackage{xcolor}
\usepackage{tcolorbox} % for boxed text
\tcbuselibrary{skins} % for better skinning support

% Increase the overall text width (optional)
\usepackage[margin=1in]{geometry}

\begin{document}

\begin{tcolorbox}[colback=gray!5!white, colframe=black!75!black, title=Figure 1. Example of the Equality of Row Rank and Column Rank, left=5pt, right=5pt, top=10pt, bottom=10pt, enlarge right by=-.5mm]
Consider the matrix $A$ defined as:
\[
A = \begin{bmatrix}
    1 & \frac{3}{2} & 0 & 0 & 0\\
    1 & 0 & \frac{1}{2} & 0 & 1\\
    0 & 1 & 0 & 1 & 1 \\
    0 & 0 & 1 & 3 & 5
\end{bmatrix}
\]

\subsection*{Non-redundant Columns}
The fourth and fifth columns of matrix $A$ can be expressed as linear combinations of the first three columns:
\begin{align*}
\begin{bmatrix}
    0 \\
    0 \\
    1 \\
    3
\end{bmatrix} &= -\frac{3}{2} \begin{bmatrix}
    1 \\
    1\\
    0 \\
    0
\end{bmatrix}
+ 1 \begin{bmatrix}
    \frac{3}{2} \\
    0\\
    1 \\
    0
\end{bmatrix}
+ 3 \begin{bmatrix}
    0 \\
    \frac{1}{2}\\
    0 \\
    1
\end{bmatrix}, \ \text{ and } \ 
\begin{bmatrix}
    0 \\
    1 \\
    1 \\
    5
\end{bmatrix} = 2 \begin{bmatrix}
    0 \\
    \frac{1}{2}\\
    0 \\
    1
\end{bmatrix}
+ 1 \begin{bmatrix}
    0 \\
    0\\
    1\\
    3
\end{bmatrix}
\end{align*}
Thus, $A$ has three non-redundant columns. 
\subsection*{Non-redundant Rows}
Similarly, the fourth row can be written in terms of the first three rows:
\[
\begin{bmatrix}
    0 & 0 & 1 & 3 & 5
\end{bmatrix}
= -2 \begin{bmatrix}
    1 & \frac{3}{2} & 0 &0 &0
\end{bmatrix}
+ 2 \begin{bmatrix}
    1 & 0 & \frac{1}{2} & 0 &1
\end{bmatrix}
+ 3 \begin{bmatrix}
    0 & 1 & 0 & 1 &1
\end{bmatrix}
\]
Thus, $A$ has three non-redundant rows. 
\subsection*{The number of non-redundant rows always equals the number of non-redundant columns}
It is a remarkable observation that this is in fact always the case for matrices of real numbers. $\it{All}$ real matrices have the same maximum number of non-redundant rows and non-redundant columns. The equality of row rank and column rank makes it clear that a matrix is more than a collection of numbers. There is a fundamental relationship between the rows and columns of a matrix. 
\end{tcolorbox}

\end{document}
